% Options for packages loaded elsewhere
\PassOptionsToPackage{unicode}{hyperref}
\PassOptionsToPackage{hyphens}{url}
%
\documentclass[
]{article}
\usepackage{amsmath,amssymb}
\usepackage{iftex}
\ifPDFTeX
  \usepackage[T1]{fontenc}
  \usepackage[utf8]{inputenc}
  \usepackage{textcomp} % provide euro and other symbols
\else % if luatex or xetex
  \usepackage{unicode-math} % this also loads fontspec
  \defaultfontfeatures{Scale=MatchLowercase}
  \defaultfontfeatures[\rmfamily]{Ligatures=TeX,Scale=1}
\fi
\usepackage{lmodern}
\ifPDFTeX\else
  % xetex/luatex font selection
\fi
% Use upquote if available, for straight quotes in verbatim environments
\IfFileExists{upquote.sty}{\usepackage{upquote}}{}
\IfFileExists{microtype.sty}{% use microtype if available
  \usepackage[]{microtype}
  \UseMicrotypeSet[protrusion]{basicmath} % disable protrusion for tt fonts
}{}
\makeatletter
\@ifundefined{KOMAClassName}{% if non-KOMA class
  \IfFileExists{parskip.sty}{%
    \usepackage{parskip}
  }{% else
    \setlength{\parindent}{0pt}
    \setlength{\parskip}{6pt plus 2pt minus 1pt}}
}{% if KOMA class
  \KOMAoptions{parskip=half}}
\makeatother
\usepackage{xcolor}
\usepackage[margin=1in]{geometry}
\usepackage{color}
\usepackage{fancyvrb}
\newcommand{\VerbBar}{|}
\newcommand{\VERB}{\Verb[commandchars=\\\{\}]}
\DefineVerbatimEnvironment{Highlighting}{Verbatim}{commandchars=\\\{\}}
% Add ',fontsize=\small' for more characters per line
\usepackage{framed}
\definecolor{shadecolor}{RGB}{248,248,248}
\newenvironment{Shaded}{\begin{snugshade}}{\end{snugshade}}
\newcommand{\AlertTok}[1]{\textcolor[rgb]{0.94,0.16,0.16}{#1}}
\newcommand{\AnnotationTok}[1]{\textcolor[rgb]{0.56,0.35,0.01}{\textbf{\textit{#1}}}}
\newcommand{\AttributeTok}[1]{\textcolor[rgb]{0.13,0.29,0.53}{#1}}
\newcommand{\BaseNTok}[1]{\textcolor[rgb]{0.00,0.00,0.81}{#1}}
\newcommand{\BuiltInTok}[1]{#1}
\newcommand{\CharTok}[1]{\textcolor[rgb]{0.31,0.60,0.02}{#1}}
\newcommand{\CommentTok}[1]{\textcolor[rgb]{0.56,0.35,0.01}{\textit{#1}}}
\newcommand{\CommentVarTok}[1]{\textcolor[rgb]{0.56,0.35,0.01}{\textbf{\textit{#1}}}}
\newcommand{\ConstantTok}[1]{\textcolor[rgb]{0.56,0.35,0.01}{#1}}
\newcommand{\ControlFlowTok}[1]{\textcolor[rgb]{0.13,0.29,0.53}{\textbf{#1}}}
\newcommand{\DataTypeTok}[1]{\textcolor[rgb]{0.13,0.29,0.53}{#1}}
\newcommand{\DecValTok}[1]{\textcolor[rgb]{0.00,0.00,0.81}{#1}}
\newcommand{\DocumentationTok}[1]{\textcolor[rgb]{0.56,0.35,0.01}{\textbf{\textit{#1}}}}
\newcommand{\ErrorTok}[1]{\textcolor[rgb]{0.64,0.00,0.00}{\textbf{#1}}}
\newcommand{\ExtensionTok}[1]{#1}
\newcommand{\FloatTok}[1]{\textcolor[rgb]{0.00,0.00,0.81}{#1}}
\newcommand{\FunctionTok}[1]{\textcolor[rgb]{0.13,0.29,0.53}{\textbf{#1}}}
\newcommand{\ImportTok}[1]{#1}
\newcommand{\InformationTok}[1]{\textcolor[rgb]{0.56,0.35,0.01}{\textbf{\textit{#1}}}}
\newcommand{\KeywordTok}[1]{\textcolor[rgb]{0.13,0.29,0.53}{\textbf{#1}}}
\newcommand{\NormalTok}[1]{#1}
\newcommand{\OperatorTok}[1]{\textcolor[rgb]{0.81,0.36,0.00}{\textbf{#1}}}
\newcommand{\OtherTok}[1]{\textcolor[rgb]{0.56,0.35,0.01}{#1}}
\newcommand{\PreprocessorTok}[1]{\textcolor[rgb]{0.56,0.35,0.01}{\textit{#1}}}
\newcommand{\RegionMarkerTok}[1]{#1}
\newcommand{\SpecialCharTok}[1]{\textcolor[rgb]{0.81,0.36,0.00}{\textbf{#1}}}
\newcommand{\SpecialStringTok}[1]{\textcolor[rgb]{0.31,0.60,0.02}{#1}}
\newcommand{\StringTok}[1]{\textcolor[rgb]{0.31,0.60,0.02}{#1}}
\newcommand{\VariableTok}[1]{\textcolor[rgb]{0.00,0.00,0.00}{#1}}
\newcommand{\VerbatimStringTok}[1]{\textcolor[rgb]{0.31,0.60,0.02}{#1}}
\newcommand{\WarningTok}[1]{\textcolor[rgb]{0.56,0.35,0.01}{\textbf{\textit{#1}}}}
\usepackage{graphicx}
\makeatletter
\def\maxwidth{\ifdim\Gin@nat@width>\linewidth\linewidth\else\Gin@nat@width\fi}
\def\maxheight{\ifdim\Gin@nat@height>\textheight\textheight\else\Gin@nat@height\fi}
\makeatother
% Scale images if necessary, so that they will not overflow the page
% margins by default, and it is still possible to overwrite the defaults
% using explicit options in \includegraphics[width, height, ...]{}
\setkeys{Gin}{width=\maxwidth,height=\maxheight,keepaspectratio}
% Set default figure placement to htbp
\makeatletter
\def\fps@figure{htbp}
\makeatother
\setlength{\emergencystretch}{3em} % prevent overfull lines
\providecommand{\tightlist}{%
  \setlength{\itemsep}{0pt}\setlength{\parskip}{0pt}}
\setcounter{secnumdepth}{-\maxdimen} % remove section numbering
\ifLuaTeX
  \usepackage{selnolig}  % disable illegal ligatures
\fi
\usepackage{bookmark}
\IfFileExists{xurl.sty}{\usepackage{xurl}}{} % add URL line breaks if available
\urlstyle{same}
\hypersetup{
  pdftitle={Customer Data Analysis},
  pdfauthor={Sahil Thorat},
  hidelinks,
  pdfcreator={LaTeX via pandoc}}

\title{Customer Data Analysis}
\author{Sahil Thorat}
\date{2024-07-25}

\begin{document}
\maketitle

\subsection{Loading and Exploring the
Data}\label{loading-and-exploring-the-data}

\begin{Shaded}
\begin{Highlighting}[]
\FunctionTok{getwd}\NormalTok{()}
\end{Highlighting}
\end{Shaded}

\begin{verbatim}
## [1] "D:/Projects/Data_Projects_AIO/Customer_Segmentation/Code"
\end{verbatim}

\begin{Shaded}
\begin{Highlighting}[]
\FunctionTok{library}\NormalTok{(readr)}
\CommentTok{\# Load the data}
\NormalTok{customer\_data}\OtherTok{=}\FunctionTok{read.csv}\NormalTok{(}\StringTok{"D:/Projects/Data\_Projects\_AIO/Customer\_Segmentation/data/Mall\_Customers.csv"}\NormalTok{)}

\FunctionTok{head}\NormalTok{(customer\_data)}
\end{Highlighting}
\end{Shaded}

\begin{verbatim}
##   CustomerID Gender Age Annual.Income..k.. Spending.Score..1.100.
## 1          1   Male  19                 15                     39
## 2          2   Male  21                 15                     81
## 3          3 Female  20                 16                      6
## 4          4 Female  23                 16                     77
## 5          5 Female  31                 17                     40
## 6          6 Female  22                 17                     76
\end{verbatim}

\begin{Shaded}
\begin{Highlighting}[]
\FunctionTok{str}\NormalTok{(customer\_data)}
\end{Highlighting}
\end{Shaded}

\begin{verbatim}
## 'data.frame':    200 obs. of  5 variables:
##  $ CustomerID            : int  1 2 3 4 5 6 7 8 9 10 ...
##  $ Gender                : chr  "Male" "Male" "Female" "Female" ...
##  $ Age                   : int  19 21 20 23 31 22 35 23 64 30 ...
##  $ Annual.Income..k..    : int  15 15 16 16 17 17 18 18 19 19 ...
##  $ Spending.Score..1.100.: int  39 81 6 77 40 76 6 94 3 72 ...
\end{verbatim}

\begin{Shaded}
\begin{Highlighting}[]
\FunctionTok{summary}\NormalTok{(customer\_data)}
\end{Highlighting}
\end{Shaded}

\begin{verbatim}
##    CustomerID        Gender               Age        Annual.Income..k..
##  Min.   :  1.00   Length:200         Min.   :18.00   Min.   : 15.00    
##  1st Qu.: 50.75   Class :character   1st Qu.:28.75   1st Qu.: 41.50    
##  Median :100.50   Mode  :character   Median :36.00   Median : 61.50    
##  Mean   :100.50                      Mean   :38.85   Mean   : 60.56    
##  3rd Qu.:150.25                      3rd Qu.:49.00   3rd Qu.: 78.00    
##  Max.   :200.00                      Max.   :70.00   Max.   :137.00    
##  Spending.Score..1.100.
##  Min.   : 1.00         
##  1st Qu.:34.75         
##  Median :50.00         
##  Mean   :50.20         
##  3rd Qu.:73.00         
##  Max.   :99.00
\end{verbatim}

\begin{Shaded}
\begin{Highlighting}[]
\FunctionTok{names}\NormalTok{(customer\_data)}
\end{Highlighting}
\end{Shaded}

\begin{verbatim}
## [1] "CustomerID"             "Gender"                 "Age"                   
## [4] "Annual.Income..k.."     "Spending.Score..1.100."
\end{verbatim}

\begin{Shaded}
\begin{Highlighting}[]
\CommentTok{\# Rename the columns}
\FunctionTok{names}\NormalTok{(customer\_data)}\OtherTok{=}\FunctionTok{c}\NormalTok{(}\StringTok{"CustomerID"}\NormalTok{, }\StringTok{"Gender"}\NormalTok{, }\StringTok{"Age"}\NormalTok{, }\StringTok{"AnnualIncome"}\NormalTok{, }\StringTok{"SpendingScore"}\NormalTok{)}

\CommentTok{\# Standard deviation for Age}
\FunctionTok{sd}\NormalTok{(customer\_data}\SpecialCharTok{$}\NormalTok{Age)}
\end{Highlighting}
\end{Shaded}

\begin{verbatim}
## [1] 13.96901
\end{verbatim}

\begin{Shaded}
\begin{Highlighting}[]
\CommentTok{\# Summary and SD statistics for Annual Income}
\FunctionTok{summary}\NormalTok{(customer\_data}\SpecialCharTok{$}\NormalTok{AnnualIncome)}
\end{Highlighting}
\end{Shaded}

\begin{verbatim}
##    Min. 1st Qu.  Median    Mean 3rd Qu.    Max. 
##   15.00   41.50   61.50   60.56   78.00  137.00
\end{verbatim}

\begin{Shaded}
\begin{Highlighting}[]
\FunctionTok{sd}\NormalTok{(customer\_data}\SpecialCharTok{$}\NormalTok{AnnualIncome)}
\end{Highlighting}
\end{Shaded}

\begin{verbatim}
## [1] 26.26472
\end{verbatim}

\begin{Shaded}
\begin{Highlighting}[]
\CommentTok{\# Summary and SD statistics for Spending Score}
\FunctionTok{summary}\NormalTok{(customer\_data}\SpecialCharTok{$}\NormalTok{SpendingScore)}
\end{Highlighting}
\end{Shaded}

\begin{verbatim}
##    Min. 1st Qu.  Median    Mean 3rd Qu.    Max. 
##    1.00   34.75   50.00   50.20   73.00   99.00
\end{verbatim}

\begin{Shaded}
\begin{Highlighting}[]
\FunctionTok{sd}\NormalTok{(customer\_data}\SpecialCharTok{$}\NormalTok{SpendingScore)}
\end{Highlighting}
\end{Shaded}

\begin{verbatim}
## [1] 25.82352
\end{verbatim}

\subsection{Visualizations}\label{visualizations}

\subsubsection{Bar Plot for Gender
Distribution}\label{bar-plot-for-gender-distribution}

\begin{Shaded}
\begin{Highlighting}[]
\CommentTok{\# Frequency table for Gender}
\NormalTok{a}\OtherTok{=}\FunctionTok{table}\NormalTok{(customer\_data}\SpecialCharTok{$}\NormalTok{Gender)}

\CommentTok{\# Barplot for gender distribution}
\NormalTok{barplot\_heights}\OtherTok{=}\FunctionTok{barplot}\NormalTok{(a, }\AttributeTok{main=}\StringTok{"Gender Comparison"}\NormalTok{,}
        \AttributeTok{ylab=}\StringTok{"Count"}\NormalTok{, }\AttributeTok{xlab=}\StringTok{"Gender"}\NormalTok{,}
        \AttributeTok{col=}\FunctionTok{rainbow}\NormalTok{(}\DecValTok{2}\NormalTok{),}
        \AttributeTok{legend=}\FunctionTok{rownames}\NormalTok{(a))}

\FunctionTok{text}\NormalTok{(barplot\_heights, a, }\AttributeTok{labels =}\NormalTok{ a, }\AttributeTok{pos =} \DecValTok{3}\NormalTok{, }\AttributeTok{cex =} \FloatTok{0.8}\NormalTok{, }\AttributeTok{col =} \StringTok{"black"}\NormalTok{)}
\end{Highlighting}
\end{Shaded}

\includegraphics{Project_code_files/figure-latex/unnamed-chunk-3-1.pdf}

\subsubsection{3D Pie Chart for Gender
Ratio}\label{d-pie-chart-for-gender-ratio}

\begin{Shaded}
\begin{Highlighting}[]
\FunctionTok{library}\NormalTok{(plotrix)}

\NormalTok{pct}\OtherTok{=}\FunctionTok{round}\NormalTok{(a }\SpecialCharTok{/} \FunctionTok{sum}\NormalTok{(a) }\SpecialCharTok{*} \DecValTok{100}\NormalTok{)}

\CommentTok{\# Labels for the pie chart}
\NormalTok{lbs}\OtherTok{=}\FunctionTok{paste}\NormalTok{(}\FunctionTok{c}\NormalTok{(}\StringTok{"Female"}\NormalTok{, }\StringTok{"Male"}\NormalTok{), }\StringTok{" "}\NormalTok{, pct, }\StringTok{"\%"}\NormalTok{, }\AttributeTok{sep=}\StringTok{" "}\NormalTok{)}
\FunctionTok{pie3D}\NormalTok{(a, }\AttributeTok{labels=}\NormalTok{lbs, }\AttributeTok{main=}\StringTok{"Pie Chart Ratio of Female Vs Male"}\NormalTok{)}
\end{Highlighting}
\end{Shaded}

\includegraphics{Project_code_files/figure-latex/unnamed-chunk-4-1.pdf}

\subsubsection{Histogram and Boxplot for
Age}\label{histogram-and-boxplot-for-age}

\begin{Shaded}
\begin{Highlighting}[]
\CommentTok{\# Histogram for Age distribution}
\FunctionTok{hist}\NormalTok{(customer\_data}\SpecialCharTok{$}\NormalTok{Age, }\AttributeTok{col=}\StringTok{"blue"}\NormalTok{,}
     \AttributeTok{main=}\StringTok{"Count of each Age Class"}\NormalTok{,}
     \AttributeTok{xlab=}\StringTok{"Age Class"}\NormalTok{, }\AttributeTok{ylab=}\StringTok{"Frequency"}\NormalTok{,}
     \AttributeTok{labels=}\ConstantTok{TRUE}\NormalTok{)}
\end{Highlighting}
\end{Shaded}

\includegraphics{Project_code_files/figure-latex/unnamed-chunk-5-1.pdf}

\begin{Shaded}
\begin{Highlighting}[]
\CommentTok{\# Boxplot for Age}
\FunctionTok{boxplot}\NormalTok{(customer\_data}\SpecialCharTok{$}\NormalTok{Age, }\AttributeTok{main=}\StringTok{"Boxplot for Descriptive Analysis of Age"}\NormalTok{)}
\end{Highlighting}
\end{Shaded}

\includegraphics{Project_code_files/figure-latex/unnamed-chunk-5-2.pdf}

\subsubsection{Annual Income Analysis}\label{annual-income-analysis}

\begin{Shaded}
\begin{Highlighting}[]
\CommentTok{\# Histogram for Annual Income}
\FunctionTok{hist}\NormalTok{(customer\_data}\SpecialCharTok{$}\NormalTok{AnnualIncome, }\AttributeTok{col=}\StringTok{"\#660033"}\NormalTok{,}
     \AttributeTok{main=}\StringTok{"Histogram Plot for Annual Income"}\NormalTok{,}
     \AttributeTok{xlab=}\StringTok{"Annual Income Class"}\NormalTok{, }\AttributeTok{ylab=}\StringTok{"Frequency"}\NormalTok{,}
     \AttributeTok{labels=}\ConstantTok{TRUE}\NormalTok{)}
\end{Highlighting}
\end{Shaded}

\includegraphics{Project_code_files/figure-latex/unnamed-chunk-6-1.pdf}

\begin{Shaded}
\begin{Highlighting}[]
\CommentTok{\# Density plot for Annual Income}
\FunctionTok{plot}\NormalTok{(}\FunctionTok{density}\NormalTok{(customer\_data}\SpecialCharTok{$}\NormalTok{AnnualIncome), }\AttributeTok{col=}\StringTok{"yellow"}\NormalTok{,}
     \AttributeTok{main=}\StringTok{"Density Plot for Annual Income"}\NormalTok{,}
     \AttributeTok{xlab=}\StringTok{"Annual Income Class"}\NormalTok{, }\AttributeTok{ylab=}\StringTok{"Density"}\NormalTok{)}

\CommentTok{\# Adding polygon to the density plot}
\FunctionTok{polygon}\NormalTok{(}\FunctionTok{density}\NormalTok{(customer\_data}\SpecialCharTok{$}\NormalTok{AnnualIncome), }\AttributeTok{col=}\StringTok{"\#ccff66"}\NormalTok{)}
\end{Highlighting}
\end{Shaded}

\includegraphics{Project_code_files/figure-latex/unnamed-chunk-6-2.pdf}

\subsubsection{Spending Score Analysis}\label{spending-score-analysis}

\begin{Shaded}
\begin{Highlighting}[]
\CommentTok{\# Boxplot for Spending Score}
\FunctionTok{boxplot}\NormalTok{(customer\_data}\SpecialCharTok{$}\NormalTok{SpendingScore, }\AttributeTok{horizontal=}\ConstantTok{TRUE}\NormalTok{,}\AttributeTok{main=}\StringTok{"BoxPlot for Descriptive Analysis of Spending Score"}\NormalTok{)}
\end{Highlighting}
\end{Shaded}

\includegraphics{Project_code_files/figure-latex/unnamed-chunk-7-1.pdf}

\begin{Shaded}
\begin{Highlighting}[]
\CommentTok{\# Histogram for Spending Score}
\FunctionTok{hist}\NormalTok{(customer\_data}\SpecialCharTok{$}\NormalTok{SpendingScore, }\AttributeTok{main=}\StringTok{"Histogram for Spending Score"}\NormalTok{,}
     \AttributeTok{xlab=}\StringTok{"Spending Score Class"}\NormalTok{, }\AttributeTok{ylab=}\StringTok{"Frequency"}\NormalTok{,}
     \AttributeTok{col=}\StringTok{"\#6600cc"}\NormalTok{, }\AttributeTok{labels=}\ConstantTok{TRUE}\NormalTok{)}
\end{Highlighting}
\end{Shaded}

\includegraphics{Project_code_files/figure-latex/unnamed-chunk-7-2.pdf}

\subsection{K-Means Clustering}\label{k-means-clustering}

\begin{Shaded}
\begin{Highlighting}[]
\CommentTok{\# Load the purrr package}
\FunctionTok{library}\NormalTok{(purrr)}
\FunctionTok{set.seed}\NormalTok{(}\DecValTok{123}\NormalTok{)}

\CommentTok{\# Function to calculate total intra{-}cluster sum of square}
\NormalTok{iss}\OtherTok{=}\ControlFlowTok{function}\NormalTok{(k) \{}
  \FunctionTok{kmeans}\NormalTok{(customer\_data[, }\DecValTok{3}\SpecialCharTok{:}\DecValTok{5}\NormalTok{], k, }\AttributeTok{iter.max=}\DecValTok{100}\NormalTok{, }\AttributeTok{nstart=}\DecValTok{100}\NormalTok{, }\AttributeTok{algorithm=}\StringTok{"Lloyd"}\NormalTok{)}\SpecialCharTok{$}\NormalTok{tot.withinss}
\NormalTok{\}}

\CommentTok{\# Define the range of k values}
\NormalTok{k.values}\OtherTok{=}\DecValTok{1}\SpecialCharTok{:}\DecValTok{10}

\CommentTok{\# Calculate the total intra{-}cluster sum of square for each k}
\NormalTok{iss\_values}\OtherTok{=}\FunctionTok{map\_dbl}\NormalTok{(k.values, iss)}

\CommentTok{\# Plot the total intra{-}cluster sum of squares against k}
\FunctionTok{plot}\NormalTok{(k.values, iss\_values, }\AttributeTok{type=}\StringTok{"b"}\NormalTok{, }\AttributeTok{pch =} \DecValTok{19}\NormalTok{, }\AttributeTok{frame =} \ConstantTok{FALSE}\NormalTok{, }
     \AttributeTok{xlab=}\StringTok{"Number of clusters K"}\NormalTok{, }\AttributeTok{ylab=}\StringTok{"Total intra{-}clusters sum of squares"}\NormalTok{)}
\end{Highlighting}
\end{Shaded}

\includegraphics{Project_code_files/figure-latex/unnamed-chunk-8-1.pdf}

\begin{Shaded}
\begin{Highlighting}[]
\FunctionTok{library}\NormalTok{(NbClust)}
\FunctionTok{library}\NormalTok{(factoextra)}
\end{Highlighting}
\end{Shaded}

\begin{verbatim}
## Warning: package 'factoextra' was built under R version 4.4.1
\end{verbatim}

\begin{verbatim}
## Loading required package: ggplot2
\end{verbatim}

\begin{verbatim}
## Welcome! Want to learn more? See two factoextra-related books at https://goo.gl/ve3WBa
\end{verbatim}

\begin{Shaded}
\begin{Highlighting}[]
\FunctionTok{fviz\_nbclust}\NormalTok{(customer\_data[,}\DecValTok{3}\SpecialCharTok{:}\DecValTok{5}\NormalTok{], kmeans, }\AttributeTok{method =} \StringTok{"wss"}\NormalTok{)}
\end{Highlighting}
\end{Shaded}

\includegraphics{Project_code_files/figure-latex/unnamed-chunk-9-1.pdf}

\begin{Shaded}
\begin{Highlighting}[]
\FunctionTok{fviz\_nbclust}\NormalTok{(customer\_data[,}\DecValTok{3}\SpecialCharTok{:}\DecValTok{5}\NormalTok{], kmeans, }\AttributeTok{method =} \StringTok{"silhouette"}\NormalTok{)}
\end{Highlighting}
\end{Shaded}

\includegraphics{Project_code_files/figure-latex/unnamed-chunk-9-2.pdf}

\subsubsection{Optimal Selection would be with 5 (wss) or 6
(silhouette).After looking with both we would be going with
5(wss).}\label{optimal-selection-would-be-with-5-wss-or-6-silhouette.after-looking-with-both-we-would-be-going-with-5wss.}

\begin{Shaded}
\begin{Highlighting}[]
\NormalTok{k6}\OtherTok{\textless{}{-}}\FunctionTok{kmeans}\NormalTok{(customer\_data[,}\DecValTok{3}\SpecialCharTok{:}\DecValTok{5}\NormalTok{],}\DecValTok{5}\NormalTok{,}\AttributeTok{iter.max=}\DecValTok{100}\NormalTok{,}\AttributeTok{nstart=}\DecValTok{50}\NormalTok{,}\AttributeTok{algorithm=}\StringTok{"Lloyd"}\NormalTok{)}
\NormalTok{k6}
\end{Highlighting}
\end{Shaded}

\begin{verbatim}
## K-means clustering with 5 clusters of sizes 39, 37, 22, 79, 23
## 
## Cluster means:
##        Age AnnualIncome SpendingScore
## 1 32.69231     86.53846      82.12821
## 2 40.32432     87.43243      18.18919
## 3 25.27273     25.72727      79.36364
## 4 43.12658     54.82278      49.83544
## 5 45.21739     26.30435      20.91304
## 
## Clustering vector:
##   [1] 5 3 5 3 5 3 5 3 5 3 5 3 5 3 5 3 5 3 5 3 5 3 5 3 5 3 5 3 5 3 5 3 5 3 5 3 5
##  [38] 3 5 3 5 3 5 4 5 3 4 4 4 4 4 4 4 4 4 4 4 4 4 4 4 4 4 4 4 4 4 4 4 4 4 4 4 4
##  [75] 4 4 4 4 4 4 4 4 4 4 4 4 4 4 4 4 4 4 4 4 4 4 4 4 4 4 4 4 4 4 4 4 4 4 4 4 4
## [112] 4 4 4 4 4 4 4 4 4 4 4 4 1 2 1 4 1 2 1 2 1 2 1 2 1 2 1 2 1 2 1 2 1 2 1 2 1
## [149] 2 1 2 1 2 1 2 1 2 1 2 1 2 1 2 1 2 1 2 1 2 1 2 1 2 1 2 1 2 1 2 1 2 1 2 1 2
## [186] 1 2 1 2 1 2 1 2 1 2 1 2 1 2 1
## 
## Within cluster sum of squares by cluster:
## [1] 13972.359 18448.865  4099.818 29909.114  8948.609
##  (between_SS / total_SS =  75.6 %)
## 
## Available components:
## 
## [1] "cluster"      "centers"      "totss"        "withinss"     "tot.withinss"
## [6] "betweenss"    "size"         "iter"         "ifault"
\end{verbatim}

\subsection{K-Means Clustering
Visualization}\label{k-means-clustering-visualization}

\begin{Shaded}
\begin{Highlighting}[]
\NormalTok{pcclust}\OtherTok{=}\FunctionTok{prcomp}\NormalTok{(customer\_data[,}\DecValTok{3}\SpecialCharTok{:}\DecValTok{5}\NormalTok{],}\AttributeTok{scale=}\ConstantTok{FALSE}\NormalTok{) }\CommentTok{\#principal component analysis}
\FunctionTok{summary}\NormalTok{(pcclust)}
\end{Highlighting}
\end{Shaded}

\begin{verbatim}
## Importance of components:
##                            PC1     PC2     PC3
## Standard deviation     26.4625 26.1597 12.9317
## Proportion of Variance  0.4512  0.4410  0.1078
## Cumulative Proportion   0.4512  0.8922  1.0000
\end{verbatim}

\begin{Shaded}
\begin{Highlighting}[]
\NormalTok{pcclust}\SpecialCharTok{$}\NormalTok{rotation[,}\DecValTok{1}\SpecialCharTok{:}\DecValTok{2}\NormalTok{]}
\end{Highlighting}
\end{Shaded}

\begin{verbatim}
##                      PC1        PC2
## Age            0.1889742 -0.1309652
## AnnualIncome  -0.5886410 -0.8083757
## SpendingScore -0.7859965  0.5739136
\end{verbatim}

\begin{Shaded}
\begin{Highlighting}[]
\FunctionTok{set.seed}\NormalTok{(}\DecValTok{1}\NormalTok{)}
\FunctionTok{ggplot}\NormalTok{(customer\_data, }\FunctionTok{aes}\NormalTok{(}\AttributeTok{x =}\NormalTok{AnnualIncome, }\AttributeTok{y =}\NormalTok{ SpendingScore)) }\SpecialCharTok{+} 
  \FunctionTok{geom\_point}\NormalTok{(}\AttributeTok{stat =} \StringTok{"identity"}\NormalTok{, }\FunctionTok{aes}\NormalTok{(}\AttributeTok{color =} \FunctionTok{as.factor}\NormalTok{(k6}\SpecialCharTok{$}\NormalTok{cluster))) }\SpecialCharTok{+}
  \FunctionTok{scale\_color\_discrete}\NormalTok{(}\AttributeTok{name=}\StringTok{" "}\NormalTok{,}
              \AttributeTok{breaks=}\FunctionTok{c}\NormalTok{(}\StringTok{"1"}\NormalTok{, }\StringTok{"2"}\NormalTok{, }\StringTok{"3"}\NormalTok{, }\StringTok{"4"}\NormalTok{, }\StringTok{"5"}\NormalTok{),}
              \AttributeTok{labels=}\FunctionTok{c}\NormalTok{(}\StringTok{"Cluster 1"}\NormalTok{, }\StringTok{"Cluster 2"}\NormalTok{, }\StringTok{"Cluster 3"}\NormalTok{, }\StringTok{"Cluster 4"}\NormalTok{, }\StringTok{"Cluster 5"}\NormalTok{)) }\SpecialCharTok{+}
  \FunctionTok{ggtitle}\NormalTok{(}\StringTok{"Mall Customers Spending Vs Income"}\NormalTok{, }\AttributeTok{subtitle =} \StringTok{"Using K{-}means Clustering"}\NormalTok{)}
\end{Highlighting}
\end{Shaded}

\includegraphics{Project_code_files/figure-latex/unnamed-chunk-12-1.pdf}

\subsubsection{Final Output}\label{final-output}

\begin{Shaded}
\begin{Highlighting}[]
\NormalTok{kCols}\OtherTok{=}\ControlFlowTok{function}\NormalTok{(vec)\{cols}\OtherTok{=}\FunctionTok{rainbow}\NormalTok{ (}\FunctionTok{length}\NormalTok{ (}\FunctionTok{unique}\NormalTok{ (vec)))}
\FunctionTok{return}\NormalTok{ (cols[}\FunctionTok{as.numeric}\NormalTok{(}\FunctionTok{as.factor}\NormalTok{(vec))])\}}

\NormalTok{digCluster}\OtherTok{\textless{}{-}}\NormalTok{k6}\SpecialCharTok{$}\NormalTok{cluster; dignm}\OtherTok{\textless{}{-}}\FunctionTok{as.character}\NormalTok{(digCluster); }\CommentTok{\# K{-}means clusters}

\FunctionTok{plot}\NormalTok{(pcclust}\SpecialCharTok{$}\NormalTok{x[,}\DecValTok{1}\SpecialCharTok{:}\DecValTok{2}\NormalTok{], }\AttributeTok{col =}\FunctionTok{kCols}\NormalTok{(digCluster),}\AttributeTok{pch =}\DecValTok{19}\NormalTok{,}\AttributeTok{xlab =}\StringTok{"K{-}means"}\NormalTok{,}\AttributeTok{ylab=}\StringTok{"classes"}\NormalTok{)}
\FunctionTok{legend}\NormalTok{(}\StringTok{"bottomleft"}\NormalTok{,}\FunctionTok{unique}\NormalTok{(dignm),}\AttributeTok{fill=}\FunctionTok{unique}\NormalTok{(}\FunctionTok{kCols}\NormalTok{(digCluster)))}
\end{Highlighting}
\end{Shaded}

\includegraphics{Project_code_files/figure-latex/unnamed-chunk-13-1.pdf}

\end{document}
